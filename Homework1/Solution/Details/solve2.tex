\begin{solution}
    \begin{enumerate}
        \item[(i)] 
        由于是全连接层,$FC_{1A}$层的输入:$$Z^{FC_{1A}}=\theta_{1a}\bm{x}+b_{1a}$$

        对应$FC_{1A}$层的激活:$$a^{FC_{1A}}=\textbf{ReLU}(Z^{FC_{1A}})$$

        % 由于是直连,$DP_{1A}$层输入即$FC_{1A}$层激活:$Z^{DP_{1A}}=a^{FC_{1A}}$;

        由于是全连接层,但受到$DP_{1A}$层dropout的影响,且$FC_{2A}$层没有激活函数,$FC_{2A}$层的输入
        即预测的$\hat{y}_a$:
        $$\hat{y}_a=Z^{FC_{2A}}=\theta_{2a}\left(a^{FC_{1A}}\odot\textbf{M}\right)+b_{2a}$$
        其中$\textbf{M}$是random mask向量,运算$\odot$是向量的逐元素乘法;

        类似$FC_{1A}$层,$FC_{1B}$层的输入:$$Z^{FC_{1B}}=\theta_{1b}\bm{x}+b_{1b}$$

        对应$FC_{1B}$层的激活:$$a^{FC_{1B}}=\textbf{ReLU}(Z^{FC_{1B}})$$

        经过$\textbf{BN}$层以及随后的逐元素加运算$\oplus$,$FC_{2B}$层的输入:
        $$Z^{FC_{2B}}=\theta_{2b}\left(\textbf{BN}_{\gamma,\beta}(a^{FC_{1B}}\right)\oplus\hat{y}_a)+b_{2b}$$
        其中$\gamma,\beta$是Batch Normalize的参数;

        对应$FC_{2B}$层的激活即预测的$\hat{y}_b$:$$\hat{y}_b=a^{FC_{2B}}=\textbf{Softmax}(Z^{FC_{2B}})$$
        
        % $FC_{1A}$层的输入:$z^{FC_{1A}}=\theta_{1a}\bm{x}+b_{1a}$;

        % $FC_{1A}$层的激活:$a^{FC_{1A}}=\textbf{ReLU}(Z^{FC_{1A}})$;

        % $DP_{1A}$层的输出:$a^{FC_{1A}}=\textbf{ReLU}(Z^{FC_{1A}})$;

        % 由于是全连接层,但受到$DP_{1A}$层dropout的影响,且$FC_{2A}$层没有激活函数,$FC_{2A}$层的输入
        % 即预测的$\hat{y}_a$:
        % $$\hat{y}_a=Z^{FC_{2A}}=\theta_{2a}a^{FC_{1A}}\odot\textbf{M}+b_{2a}$$
        % 其中$\textbf{M}$是random mask向量,运算$\odot$是向量的逐元素乘法;

        % 类似$FC_{1A}$层,$FC_{1B}$层的输入:$$Z^{FC_{1B}}=\theta_{1b}\bm{x}+b_{1b}$$

        % 对应$FC_{1B}$层的激活:$$a^{FC_{1B}}=\textbf{ReLU}(Z^{FC_{1B}})$$

        % 经过$\textbf{BN}$层以及随后的逐元素加运算$\oplus$,$FC_{2B}$层的输入:
        % $$Z^{FC_{2B}}=\theta_{2b}\left(\textbf{BN}_{\gamma,\beta}(a^{FC_{1B}}\right)\oplus\hat{y}_a)+b_{2b}$$
        % 其中$\gamma,\beta$是Batch Normalize的参数;

        % 对应$FC_{2B}$层的激活即预测的$\hat{y}_b$:$$\hat{y}_b=a^{FC_{2B}}=\textbf{Softmax}(Z^{FC_{2B}})$$
        
        \item[(ii)] 损失函数$L$:
        $$L\left(\boldsymbol{x}, y_{a}, y_{b} ; \theta\right)=\frac{1}{m} \sum_{i=1}^{m}\left[\frac{1}{2}\left\|\left(\widehat{y}_{a i}-y_{a i}\right)\right\|_{2}^{2}-\sum_{j=1}^{n_{y b}} y_{b i}^{j} \log \left(\widehat{y}_{b i}^{j}\right)\right]$$

        $FC_{2B}$的残余$\delta_{FC_{2B}}$:
        \begin{align*}
            \delta_{FC_{2B}} = \frac{\partial L}{\partial z^{FC_{2B}}} = \frac{\partial L}{\partial a^{FC_{2B}}}\cdot\frac{\partial a^{FC_{2B}}}{\partial z^{FC_{2B}}}
        \end{align*}
        前者
        \begin{align*}
            \frac{\partial L}{\partial a^{FC_{2B}}}=\frac{\partial L}{\partial \hat{y}_b} = -\frac{1}{m}\sum_{i=1}^{m} \frac{y_{bi}}{\hat{y}_{bi}}
            % \begin{cases}
            %     0& \hat{y}_b=y_b
            % \end{cases}
        \end{align*}
        后者由1.1(iii)求得,代入:
        \begin{align*}
            \delta_{FC_{2B}} &= -\sum_{i=1}^{m} \sum_{j=1}^{n_{yb}}\frac{y_{bi}^j}{\hat{y}_{bi}^j}\cdot\frac{\partial a^{FC_{2B}}}{\partial z^{FC_{2B}}}\\
            &= \frac{1}{m}\sum_{i=1}^{m}\left((-\frac{y_{bi}^k}{\hat{y}_{bi}^k})\hat{y}_{bi}^k(1-\hat{y}_{bi}^k)+\sum_{j\neq k}\frac{y_{bi}^j}{\hat{y}_{bi}^j}\left(\hat{y}_{bi}^j\right)^2\right)\\
            &= \frac{1}{m}\sum_{i=1}^{m}\left(\hat{y}_{bi}-y_{bi}\right)
        \end{align*}
        对于$\theta_{2b}$的梯度:
        \begin{align*}
            \frac{\partial L}{\partial \theta_{2b}} &= \delta_{FC_{2B}} \cdot \left(\frac{\partial z^{FC_{2B}}}{\partial \theta_{2b}}\right)^T \\
            &=\frac{1}{m}\sum_{i=1}^{m}\left(\hat{y}_{bi}-y_{bi}\right)\left(\textbf{BN}_{\gamma,\beta}(a^{FC_{1B}})\oplus\hat{y}_a\right)^T
        \end{align*}

        对于Batch Normalize中$\gamma$的梯度:
        \begin{align*}
            \frac{\partial L}{\partial\gamma} &= \left(\frac{\partial \textbf{BN}_{\gamma,\beta}(a^{FC_{1B}})}{\partial \gamma}\right)^T \cdot \left(\frac{\partial z^{FC_{2B}}}{\partial\textbf{BN}_{\gamma,\beta}(a^{FC_{1B}})}\right)^T \cdot\frac{\partial L}{\partial z^{FC_{2B}}}\\
            &= \left(\hat{a}^{FC_{1B}}\right)^T\left(\theta_{2b}\right)^T\frac{1}{m}\sum_{i=1}^{m}\left(\hat{y}_{bi}-y_{bi}\right)
        \end{align*}
        其中,$\hat{a}^{FC_{1B}}$是对$a^{FC_{1B}}$的正则化。

        对于Batch Normalize中$\beta$的梯度:
        \begin{align*}
            \frac{\partial L}{\partial\beta} &= \left(\frac{\partial \textbf{BN}_{\gamma,\beta}(a^{FC_{1B}})}{\partial \beta}\right)^T \cdot \left(\frac{\partial z^{FC_{2B}}}{\partial\textbf{BN}_{\gamma,\beta}(a^{FC_{1B}})}\right)^T \cdot\frac{\partial L}{\partial z^{FC_{2B}}}\\
            &= \sum\left(\theta_{2b}\right)^T\frac{1}{m}\sum_{i=1}^{m}\left(\hat{y}_{bi}-y_{bi}\right)
        \end{align*}
        % 对于$\theta_{2b}$的梯度:
        % \begin{align*}
        %     \frac{\partial L}{\partial \theta_{2b}} &= \frac{\partial L}{\partial \hat{y}_b} \cdot \frac{\partial \hat{y}_b}{\partial \theta_{2b}} = \frac{\partial L}{\partial \hat{y}_b} \cdot \frac{\partial Z^{FC_{2B}}}{\partial \theta_{2b}}\cdot \textbf{Softmax}'(Z^{FC_{2B}})\\
        %     &= \frac{\partial L}{\partial \hat{y}_b} \cdot \left(\textbf{BN}_{\gamma,\beta}(a^{FC_{1B}})\oplus\hat{y}_a\right) \cdot \textbf{Softmax}'(Z^{FC_{2B}})
        % \end{align*}
        % 其中$\textbf{Softmax}'$可由1.1(iii)求得。

        $FC_{1B}$的残余$\delta_{FC_{1B}}$:
        \begin{align*}
            \delta_{FC_{1B}} &= \frac{\partial L}{\partial z^{FC_{1B}}} 
            =\left(\frac{\partial z^{FC_{2B}}}{\partial a^{FC_{1B}}}\right)^T \cdot \delta_{FC_{2B}} \cdot \frac{\partial a^{FC_{1B}}}{\partial z^{FC_{1B}}}\\
            &=  \left(\theta_{2b}  \textbf{BN}'_{\gamma,\beta}(a^{FC_{1B}})\right)^T \delta_{FC_{2B}} \odot \textbf{ReLU}'(z^{FC_{1B}})
            % &= \frac{1}{m} \sum_{i=1}^m \left(\hat{y}_a-y_a\right) - \frac{\partial y_{b i}^{j} \log \left(\widehat{y}_{b i}^{j}\right)}{\partial z^{FC_{2B}}}\cdot\frac{\partial z^{FC_{2B}}}{\partial\hat{y}_a}\\
            % &= \frac{1}{m} \sum_{i=1}^m \left(\left(\hat{y}_a-y_a\right) + \left(\theta_{2b}\right)^T\left(\hat{y}_{bi}-y_{bi}\right)\right)
        \end{align*}
        其中,$\frac{\partial\textbf{BN}_j}{\partial x_i}$满足:
        {\small
        \begin{align*}
            &\frac{\partial\textbf{BN}_j}{\partial x_i} = \frac{\partial \frac{x_{j}-\mu}{\sqrt{\sigma^{2}+\epsilon}}}{\partial x_{i}}\\
            &=
            \begin{cases}
                -\frac{1}{m}\left(\sigma^{2}+\epsilon\right)^{-1 / 2}-\frac{1}{m}\left(\sigma^{2}+\epsilon\right)^{-3 / 2}\left(x_{i}-\mu\right)\left(x_{j}-\mu\right)& i\neq j\\
                \left(1-\frac{1}{m}\right)\left(\sigma^{2}+\epsilon\right)^{-1 / 2}-\frac{1}{m}\left(\sigma^{2}+\epsilon\right)^{-3 / 2}\left(x_{i}-\mu\right)\left(x_{j}-\mu\right)& i=j
            \end{cases}
        \end{align*}}
        
        对于$\theta_{1b}$的梯度:
        \begin{align*}
            \frac{\partial L}{\partial \theta_{1b}} &= \delta_{FC_{1B}} \cdot \left(\frac{\partial z^{FC_{1B}}}{\partial \theta_{1b}}\right)^T \\
            &=\delta_{FC_{1B}}\bm{x}^T
        \end{align*}

        $FC_{2A}$的残余$\delta_{FC_{2A}}$:
        \begin{align*}
            \delta_{FC_{2A}} &= \frac{\partial L}{\partial z^{FC_{2A}}} = \frac{\partial L}{\partial \hat{y}_a} \\
            &= \frac{1}{m} \sum_{i=1}^m \left(\hat{y}_a-y_a\right) - \frac{\partial y_{b i}^{j} \log \left(\widehat{y}_{b i}^{j}\right)}{\partial z^{FC_{2B}}}\cdot\frac{\partial z^{FC_{2B}}}{\partial\hat{y}_a}\\
            &= \frac{1}{m} \sum_{i=1}^m \left(\left(\hat{y}_a-y_a\right) + \left(\theta_{2b}\right)^T\left(\hat{y}_{bi}-y_{bi}\right)\right)
        \end{align*}
        对于$\theta_{2a}$的梯度:
        \begin{align*}
            \frac{\partial L}{\partial \theta_{2a}} &= \delta_{FC_{2A}} \cdot \left(\frac{\partial z^{FC_{2A}}}{\partial \theta_{2a}}\right)^T \\
            &=\left(\frac{1}{m} \sum_{i=1}^m \left(\hat{y}_a-y_a\right) + \left(\theta_{2b}\right)^T\left(\hat{y}_{bi}-y_{bi}\right)\right)\bm{x}^T
        \end{align*}

        $FC_{1A}$的残余$\delta_{FC_{1A}}$:
        \begin{align*}
            \delta_{FC_{1A}} &= \frac{\partial L}{\partial z^{FC_{1A}}} 
            =\left(\frac{\partial z^{FC_{2A}}}{\partial a^{FC_{1A}}}\right)^T \cdot \delta_{FC_{2A}} \cdot \frac{\partial a^{FC_{1A}}}{\partial z^{FC_{1A}}}\\
            &=  (\theta_{2a})^T \delta_{FC_{2A}} \odot \textbf{M} \odot \textbf{ReLU}'(z^{FC_{1A}})
            % &= \frac{1}{m} \sum_{i=1}^m \left(\hat{y}_a-y_a\right) - \frac{\partial y_{b i}^{j} \log \left(\widehat{y}_{b i}^{j}\right)}{\partial z^{FC_{2B}}}\cdot\frac{\partial z^{FC_{2B}}}{\partial\hat{y}_a}\\
            % &= \frac{1}{m} \sum_{i=1}^m \left(\left(\hat{y}_a-y_a\right) + \left(\theta_{2b}\right)^T\left(\hat{y}_{bi}-y_{bi}\right)\right)
        \end{align*}
        其中,$\odot$是逐元素乘法。

        对于$\theta_{1a}$的梯度:
        \begin{align*}
            \frac{\partial L}{\partial \theta_{1a}} &= \delta_{FC_{1A}} \cdot \left(\frac{\partial z^{FC_{1A}}}{\partial \theta_{1a}}\right)^T \\
            &= \left((\theta_{2a})^T \delta_{FC_{2A}} \odot \textbf{M} \odot \textbf{ReLU}'(z^{FC_{1A}})\right)\bm{x}^T
        \end{align*}

        % 对于$\theta_{1b}$的梯度:
        % \begin{align*}
        %     \frac{\partial L}{\partial \theta_{1b}} &= \frac{\partial L}{\partial Z^{BN_{1B}}} \cdot \frac{\partial Z^{BN_{1B}}}{\partial \theta_{1b}} \\
        %     &= \frac{\partial L}{\partial Z^{BN_{1B}}} \cdot \frac{\partial Z^{FC_{1B}}}{\partial \theta_{1b}}\cdot \textbf{ReLU}'(Z^{FC_{1B}})\\
        %     &= \delta^{BN_{1B}} \cdot \bm{x}\cdot \textbf{ReLU}'(Z^{FC_{1B}})
        %     % &= \frac{\partial L}{\partial \hat{y}_b} \cdot \left(\textbf{BN}_{\gamma,\beta}(a^{FC_{1B}})\oplus\hat{y}_a\right) \cdot \textbf{Softmax}'(Z^{FC_{2B}})
        % \end{align*}
        % 其中,$BN_{1B}$层的残差$\delta^{BN_{1B}}$,满足:
        % \begin{align*}
        %     \delta^{BN_{1B}} &= \frac{\partial L}{\partial Z^{BN_{1B}}} = \frac{\partial L}{\partial \hat{y}_b} \cdot \frac{\partial \hat{y}_b}{\partial a^{BN_{1B}}} \cdot \frac{\partial a^{BN_{1B}}}{\partial Z^{BN_{1B}}} \\
        %     &= \frac{\partial L}{\partial \hat{y}_b} \cdot \frac{\partial Z^{FC_{2B}}}{\partial a^{BN_{1B}}}\cdot \textbf{Softmax}'(Z^{FC_{2B}})\cdot \textbf{BN}'(Z^{BN_{1B}})\\
        %     &= \frac{\partial L}{\partial \hat{y}_b} \cdot \theta_{2b} \cdot \textbf{Softmax}'(Z^{FC_{2B}})\cdot \textbf{BN}'(Z^{BN_{1B}})
        % \end{align*}

    \end{enumerate}
\end{solution}