\begin{solution}
    \begin{enumerate}
        \item[(i)] 由于是全连接层,$FC_{1A}$层的输入:$$Z^{FC_{1A}}=\theta_{1a}\bm{x}+b_{1a}$$

        对应$FC_{1A}$层的激活:$$a^{FC_{1A}}=\textbf{ReLU}(Z^{FC_{1A}})$$

        % 由于是直连,$DP_{1A}$层输入即$FC_{1A}$层激活:$Z^{DP_{1A}}=a^{FC_{1A}}$;

        由于是全连接层,但受到$DP_{1A}$层dropout的影响,且$FC_{2A}$层没有激活函数,$FC_{2A}$层的输入
        即预测的$\hat{y}_a$:
        $$\hat{y}_a=Z^{FC_{2A}}=\theta_{2a}a^{FC_{1A}}*\textbf{M}+b_{2a}$$
        其中$\textbf{M}$是random mask向量,运算$*$是向量的逐元素乘法;

        类似$FC_{1A}$层,$FC_{1B}$层的输入:$$Z^{FC_{1B}}=\theta_{1b}\bm{x}+b_{1b}$$

        对应$FC_{1B}$层的激活:$$a^{FC_{1B}}=\textbf{ReLU}(Z^{FC_{1B}})$$

        经过$\textbf{BN}$层以及随后的逐元素加运算$\bigoplus$,$FC_{2B}$层的输入:
        $$Z^{FC_{2B}}=\theta_{2b}(\textbf{BN}_{\gamma,\beta}(a^{FC_{1B}})\bigoplus\hat{y}_a)+b_{2b}$$
        其中$\gamma,\beta$是Bitch Normalize的参数;

        对应$FC_{2B}$层的激活即预测的$\hat{y}_b$:$$\hat{y}_b=a^{FC_{2B}}=\textbf{Softmax}(Z^{FC_{2B}})$$
        \item[(ii)] 
    \end{enumerate}
\end{solution}